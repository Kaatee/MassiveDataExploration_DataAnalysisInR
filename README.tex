\documentclass[]{article}
\usepackage{lmodern}
\usepackage{amssymb,amsmath}
\usepackage{ifxetex,ifluatex}
\usepackage{fixltx2e} % provides \textsubscript
\ifnum 0\ifxetex 1\fi\ifluatex 1\fi=0 % if pdftex
  \usepackage[T1]{fontenc}
  \usepackage[utf8]{inputenc}
\else % if luatex or xelatex
  \ifxetex
    \usepackage{mathspec}
  \else
    \usepackage{fontspec}
  \fi
  \defaultfontfeatures{Ligatures=TeX,Scale=MatchLowercase}
\fi
% use upquote if available, for straight quotes in verbatim environments
\IfFileExists{upquote.sty}{\usepackage{upquote}}{}
% use microtype if available
\IfFileExists{microtype.sty}{%
\usepackage{microtype}
\UseMicrotypeSet[protrusion]{basicmath} % disable protrusion for tt fonts
}{}
\usepackage[margin=1in]{geometry}
\usepackage{hyperref}
\hypersetup{unicode=true,
            pdftitle={Eksploracja Masywnych Danych - projekt 1 - Analiza danych},
            pdfauthor={Katarzyna Jóźwiak 127237, Piotr Pawlaczyk 127245},
            pdfborder={0 0 0},
            breaklinks=true}
\urlstyle{same}  % don't use monospace font for urls
\usepackage{color}
\usepackage{fancyvrb}
\newcommand{\VerbBar}{|}
\newcommand{\VERB}{\Verb[commandchars=\\\{\}]}
\DefineVerbatimEnvironment{Highlighting}{Verbatim}{commandchars=\\\{\}}
% Add ',fontsize=\small' for more characters per line
\usepackage{framed}
\definecolor{shadecolor}{RGB}{248,248,248}
\newenvironment{Shaded}{\begin{snugshade}}{\end{snugshade}}
\newcommand{\AlertTok}[1]{\textcolor[rgb]{0.94,0.16,0.16}{#1}}
\newcommand{\AnnotationTok}[1]{\textcolor[rgb]{0.56,0.35,0.01}{\textbf{\textit{#1}}}}
\newcommand{\AttributeTok}[1]{\textcolor[rgb]{0.77,0.63,0.00}{#1}}
\newcommand{\BaseNTok}[1]{\textcolor[rgb]{0.00,0.00,0.81}{#1}}
\newcommand{\BuiltInTok}[1]{#1}
\newcommand{\CharTok}[1]{\textcolor[rgb]{0.31,0.60,0.02}{#1}}
\newcommand{\CommentTok}[1]{\textcolor[rgb]{0.56,0.35,0.01}{\textit{#1}}}
\newcommand{\CommentVarTok}[1]{\textcolor[rgb]{0.56,0.35,0.01}{\textbf{\textit{#1}}}}
\newcommand{\ConstantTok}[1]{\textcolor[rgb]{0.00,0.00,0.00}{#1}}
\newcommand{\ControlFlowTok}[1]{\textcolor[rgb]{0.13,0.29,0.53}{\textbf{#1}}}
\newcommand{\DataTypeTok}[1]{\textcolor[rgb]{0.13,0.29,0.53}{#1}}
\newcommand{\DecValTok}[1]{\textcolor[rgb]{0.00,0.00,0.81}{#1}}
\newcommand{\DocumentationTok}[1]{\textcolor[rgb]{0.56,0.35,0.01}{\textbf{\textit{#1}}}}
\newcommand{\ErrorTok}[1]{\textcolor[rgb]{0.64,0.00,0.00}{\textbf{#1}}}
\newcommand{\ExtensionTok}[1]{#1}
\newcommand{\FloatTok}[1]{\textcolor[rgb]{0.00,0.00,0.81}{#1}}
\newcommand{\FunctionTok}[1]{\textcolor[rgb]{0.00,0.00,0.00}{#1}}
\newcommand{\ImportTok}[1]{#1}
\newcommand{\InformationTok}[1]{\textcolor[rgb]{0.56,0.35,0.01}{\textbf{\textit{#1}}}}
\newcommand{\KeywordTok}[1]{\textcolor[rgb]{0.13,0.29,0.53}{\textbf{#1}}}
\newcommand{\NormalTok}[1]{#1}
\newcommand{\OperatorTok}[1]{\textcolor[rgb]{0.81,0.36,0.00}{\textbf{#1}}}
\newcommand{\OtherTok}[1]{\textcolor[rgb]{0.56,0.35,0.01}{#1}}
\newcommand{\PreprocessorTok}[1]{\textcolor[rgb]{0.56,0.35,0.01}{\textit{#1}}}
\newcommand{\RegionMarkerTok}[1]{#1}
\newcommand{\SpecialCharTok}[1]{\textcolor[rgb]{0.00,0.00,0.00}{#1}}
\newcommand{\SpecialStringTok}[1]{\textcolor[rgb]{0.31,0.60,0.02}{#1}}
\newcommand{\StringTok}[1]{\textcolor[rgb]{0.31,0.60,0.02}{#1}}
\newcommand{\VariableTok}[1]{\textcolor[rgb]{0.00,0.00,0.00}{#1}}
\newcommand{\VerbatimStringTok}[1]{\textcolor[rgb]{0.31,0.60,0.02}{#1}}
\newcommand{\WarningTok}[1]{\textcolor[rgb]{0.56,0.35,0.01}{\textbf{\textit{#1}}}}
\usepackage{graphicx,grffile}
\makeatletter
\def\maxwidth{\ifdim\Gin@nat@width>\linewidth\linewidth\else\Gin@nat@width\fi}
\def\maxheight{\ifdim\Gin@nat@height>\textheight\textheight\else\Gin@nat@height\fi}
\makeatother
% Scale images if necessary, so that they will not overflow the page
% margins by default, and it is still possible to overwrite the defaults
% using explicit options in \includegraphics[width, height, ...]{}
\setkeys{Gin}{width=\maxwidth,height=\maxheight,keepaspectratio}
\IfFileExists{parskip.sty}{%
\usepackage{parskip}
}{% else
\setlength{\parindent}{0pt}
\setlength{\parskip}{6pt plus 2pt minus 1pt}
}
\setlength{\emergencystretch}{3em}  % prevent overfull lines
\providecommand{\tightlist}{%
  \setlength{\itemsep}{0pt}\setlength{\parskip}{0pt}}
\setcounter{secnumdepth}{0}
% Redefines (sub)paragraphs to behave more like sections
\ifx\paragraph\undefined\else
\let\oldparagraph\paragraph
\renewcommand{\paragraph}[1]{\oldparagraph{#1}\mbox{}}
\fi
\ifx\subparagraph\undefined\else
\let\oldsubparagraph\subparagraph
\renewcommand{\subparagraph}[1]{\oldsubparagraph{#1}\mbox{}}
\fi

%%% Use protect on footnotes to avoid problems with footnotes in titles
\let\rmarkdownfootnote\footnote%
\def\footnote{\protect\rmarkdownfootnote}

%%% Change title format to be more compact
\usepackage{titling}

% Create subtitle command for use in maketitle
\providecommand{\subtitle}[1]{
  \posttitle{
    \begin{center}\large#1\end{center}
    }
}

\setlength{\droptitle}{-2em}

  \title{Eksploracja Masywnych Danych - projekt 1 - Analiza danych}
    \pretitle{\vspace{\droptitle}\centering\huge}
  \posttitle{\par}
    \author{Katarzyna Jóźwiak 127237, Piotr Pawlaczyk 127245}
    \preauthor{\centering\large\emph}
  \postauthor{\par}
    \date{}
    \predate{}\postdate{}
  

\begin{document}
\maketitle

data wygenerowania: `2019-październik-24'

\hypertarget{spis-treux15bci}{%
\section{Spis treści}\label{spis-treux15bci}}

\begin{enumerate}
\def\labelenumi{\arabic{enumi}.}
\tightlist
\item
  \protect\hyperlink{summary}{Podsumowanie badań}
\item
  \protect\hyperlink{librarys}{Wykorzystane biblioteki}
\item
  \protect\hyperlink{xxx}{XXX}
\item
  \protect\hyperlink{readDataFromFile}{Wczytywanie danych z pliku}
\item
  \protect\hyperlink{missingData}{Brakujące dane}
\item
  \protect\hyperlink{statistics}{Rozmiar zbioru i statystyki}
\item
  \protect\hyperlink{analisis}{Szczegółowa analiza zbiorów wartości}
\item
  \protect\hyperlink{correlation}{Korelacja między zmiennymi}
\item
  \protect\hyperlink{animation}{Zmiana rozmiaru śledzia w czasie}
\item
  \protect\hyperlink{prediction}{Przewidywanie rozmiaru śledzia}
\item
  \protect\hyperlink{bestModelAnalisis}{Analiza ważnośći atrybutów
  najlepszego znalezionego modelu regresji}
\end{enumerate}

\hypertarget{podsumowanie-badaux144}{%
\subsection{Podsumowanie badań }\label{podsumowanie-badaux144}}

TODO: rozdział podsumowujący całą analizę

\hypertarget{wykorzystane-biblioteki}{%
\subsection{Wykorzystane biblioteki }\label{wykorzystane-biblioteki}}

Kod wyliczający wykorzystane biblioteki.

\hypertarget{xxx}{%
\subsection{XXX }\label{xxx}}

Kod zapewniający powtarzalność wyników przy każdym uruchomieniu raportu
na tych samych danych.

\hypertarget{wczytywanie-danych-z-pliku}{%
\subsection{Wczytywanie danych z pliku
}\label{wczytywanie-danych-z-pliku}}

\begin{Shaded}
\begin{Highlighting}[]
\NormalTok{data <-}\StringTok{ }\KeywordTok{read.table}\NormalTok{(}\StringTok{"sledzie.csv"}\NormalTok{, }\DataTypeTok{header=}\OtherTok{TRUE}\NormalTok{, }\DataTypeTok{sep=}\StringTok{","}\NormalTok{)}
\KeywordTok{head}\NormalTok{(data)}
\end{Highlighting}
\end{Shaded}

\begin{verbatim}
##   X length   cfin1   cfin2   chel1    chel2   lcop1    lcop2  fbar   recr
## 1 0   23.0 0.02778 0.27785 2.46875        ? 2.54787 26.35881 0.356 482831
## 2 1   22.5 0.02778 0.27785 2.46875 21.43548 2.54787 26.35881 0.356 482831
## 3 2   25.0 0.02778 0.27785 2.46875 21.43548 2.54787 26.35881 0.356 482831
## 4 3   25.5 0.02778 0.27785 2.46875 21.43548 2.54787 26.35881 0.356 482831
## 5 4   24.0 0.02778 0.27785 2.46875 21.43548 2.54787 26.35881 0.356 482831
## 6 5   22.0 0.02778 0.27785 2.46875 21.43548 2.54787        ? 0.356 482831
##        cumf   totaln           sst      sal xmonth nao
## 1 0.3059879 267380.8 14.3069330186 35.51234      7 2.8
## 2 0.3059879 267380.8 14.3069330186 35.51234      7 2.8
## 3 0.3059879 267380.8 14.3069330186 35.51234      7 2.8
## 4 0.3059879 267380.8 14.3069330186 35.51234      7 2.8
## 5 0.3059879 267380.8 14.3069330186 35.51234      7 2.8
## 6 0.3059879 267380.8 14.3069330186 35.51234      7 2.8
\end{verbatim}

\hypertarget{brakujux105ce-dane}{%
\subsection{Brakujące dane }\label{brakujux105ce-dane}}

Kod przetwarzający brakujące dane.

\hypertarget{rozmiar-zbioru-i-statystyki}{%
\subsection{Rozmiar zbioru i statystyki
}\label{rozmiar-zbioru-i-statystyki}}

Sekcja podsumowująca rozmiar zbioru i podstawowe statystyki.

\hypertarget{szczeguxf3ux142owa-analiza-zbioruxf3w-wartoux15bci}{%
\subsection{Szczegółowa analiza zbiorów wartości
}\label{szczeguxf3ux142owa-analiza-zbioruxf3w-wartoux15bci}}

Szczegółowa analiza wartości atrybutów (np. poprzez prezentację
rozkładów wartości).

\hypertarget{korelacja-miux119dzy-zmiennymi}{%
\subsection{Korelacja między zmiennymi
}\label{korelacja-miux119dzy-zmiennymi}}

Sekcja sprawdzająca korelacje między zmiennymi; sekcja ta powinna
zawierać jakąś formę graficznej prezentacji korelacji.

\hypertarget{zmiana-rozmiaru-ux15bledzia-w-czasie}{%
\subsection{Zmiana rozmiaru śledzia w czasie
}\label{zmiana-rozmiaru-ux15bledzia-w-czasie}}

Interaktywny wykres lub animację prezentującą zmianę rozmiaru śledzi w
czasie.

\hypertarget{przewidywanie-rozmiaru-ux15bledzia}{%
\subsection{Przewidywanie rozmiaru śledzia
}\label{przewidywanie-rozmiaru-ux15bledzia}}

Sekcję próbującą stworzyć regresor przewidujący rozmiar śledzia (w tej
sekcji należy wykorzystać wiedzę z pozostałych punktów oraz wykonać
dodatkowe czynności, które mogą poprawić trafność predykcji); dobór
parametrów modelu oraz oszacowanie jego skuteczności powinny zostać
wykonane za pomocą techniki podziału zbioru na dane uczące, walidujące i
testowe; trafność regresji powinna zostać oszacowana na podstawie miar
R2 i RMSE.

\hypertarget{analiza-waux17cnoux15bux107i-atrybutuxf3w-najlepszego-znalezionego-modelu-regresji}{%
\subsection{Analiza ważnośći atrybutów najlepszego znalezionego modelu
regresji
}\label{analiza-waux17cnoux15bux107i-atrybutuxf3w-najlepszego-znalezionego-modelu-regresji}}

Analiza ważności atrybutów najlepszego znalezionego modelu regresji.
Analiza ważności atrybutów powinna stanowić próbę odpowiedzi na pytanie:
co sprawia, że rozmiar śledzi zaczął w pewnym momencie maleć.


\end{document}
